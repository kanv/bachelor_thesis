%%%%%%%%%%%%%%%%%%%%%%%%%%%%%%%%%%%%%%%%%%%%%%%%%%%%%%%%%%%%%%%%%%%%%%%%%%%%%%%%
\conclusion
%%%%%%%%%%%%%%%%%%%%%%%%%%%%%%%%%%%%%%%%%%%%%%%%%%%%%%%%%%%%%%%%%%%%%%%%%%%%%%%%

В ходе данной выпускной квалификационной работы было реализовано веб-приложение, реализующее проверку моделей с помощью верификатора NuSMV. были рассмотрены существующие решения в области графических интерфейсов пользователя для верификаторов, проанализированы преимущества и недостатки. Также была спроектирована архитектура приложения, обоснован выбор используемых в проекте технологий. С помощью JavaScript библиотеки React были реализованы основные элементы управления, библиотека Redux облегчила управление потоком данных в клиентской части приложения. Было разработано поле графического построения моделей с помощью библиотеки D3 и технологии SVG. Серверная часть приложения поддерживает обработку нескольких клиентов, также был реализован API, с помощью которого происходит взаимодействие клиентов и сервера.

Данный проект можно использовать в академических курсах с целью ознакомления с формальной верификацией систем. Использование автоматического перевода модели из графического представления в текстовое описание на языке SMV, а также инспекторов переменных состояния и спецификаций, предоставляют возможность верификации модели без необходимости изучения синтаксиса SMV. При доработке проекта возможно более широкое использование в промышленных масштабах.

Тем не менее, у проекта есть проблемы, решение которых планируется в будущем. Приложение поддерживает только самые примитивные типы данных: перечисления, булевы переменные и массивы. Динамически сгенерированный код недоступен для правки, т.к. для этого необходима функция обратной генерации (из текстового представления в графическое). Также не реализована интерпретация ответа от сервера; в данном контексте наиболее интересной функцией является визуализация контрпримера. Также планируется расширить возможности приложения аутентификацией пользователей и возможностью сохранять разработанные модели в облачном сервисе или непосредственно на сервере.