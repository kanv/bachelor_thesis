\chapter{Тестирование системы}

\section{Тестирование клиентской части}

В общем случае, тестирование графических интерфейсов пользователя производится ручным методом. Однако для приложений, написанных с помощью библиотеки React, удобно использовать в качестве инструмента тестирования Jest. Jest реализует метод тестирования с помощью слепков, также поддерживается модульное тестирование с возможностью исключения зависимостей методом подстановки заглушек. Использование библиотеки рассмотрим на примере компонента StateItem, представляющего собой элемент массива переменных состояния. Создадим рендер данного элемента, используя в качестве функций обратного вызова заглушки \textit{jest.fn()}. Далле создадим два слепка рендеров: до вызова метода \textit{onChange()}и после (листинг \ref{lst:jest_test}). Результат выполнения теста представлен в листинге \ref{lst:jest}.

\begin{lstlisting}[language=Java, 
label=lst:jest_test, 
caption={Фрагмент теста для компонента StateItem.}]
const state = {stateName: 'test', type: 'bool', value: 'false'},
	vertex = {},
	enums = [],
	removeState = jest.fn(),
	changeSelect = jest.fn(),
	changeSelectArr = jest.fn();

const component = renderer.create(
<StateItem
	state={state} vertex={vertex}
	enums={enums} removeState={removeState}
	changeSelect={changeSelect}
	changeSelectArr={changeSelectArr} />
);

let tree = component.toJSON();
expect(tree).toMatchSnapshot();
tree.props.onChange();
tree = component.toJSON();
expect(tree).toMatchSnapshot();
\end{lstlisting}

\begin{lstlisting}[language=Python, 
label=lst:jest, 
caption={Фрагмент вывода сисетмы тестирования Jest.}]
> jest

FAIL  js\__tests__\StateItem-test.js
+ Select value change test

TypeError: tree.props.onChange is not a function

at Object.<anonymous> (js/__tests__/StateItem-test.js:28:16)
at Promise.resolve.then.el (node_modules/p-map/index.js:42:16)

- Select value change test (18ms)

Test Suites: 1 failed, 1 total
Tests:       1 failed, 1 total
Snapshots:   1 passed, 1 total
\end{lstlisting}

Помимо тестирования с помощью слепков, в ходе работы использовалось ручное тестирование. Было реализовано несколько разных сценариев использования.

\begin{itemize}
	\item Проверка соответствия графически введенной модели текстовому описанию;
	\item Проверка поведения программы при количестве переменных состояния равному 50;
	\item Проверка поведения программы при количестве спецификаций равному 50;
	\item Проверка поведения программы при количестве состояний равному 50 и количеству переходов равному 50;
	\item Тестирование вводимых в формы названий переменных состояния;
	\item Проверка корректности отображения модели при случайных кликах;
	\item Проверка возможности добавления переменных с одинаковым именем;
\end{itemize}

В результате тестирования выявился ряд дефектов, например, определенный набор кликов по полю построения проекта порождал неверное отображение SVG элементов. Данный дефект был локализован и устранен.

Также было проведено тестирование функции трансляции объектного представления модели в текстовую. Производилось оно путем создания графического представления и оценки генерируемого кода на соответствие данной модели. 

\section{Тестирование серверной части}

Сервер реализует следующую функциональность: обработка клиентских запросов, запуск процесса NuSMV, создание файла SMV, выдача клиенту ответа. Тестирование сервера преимущественно сводится к тестированию REST-сервиса. Для этого были использованы различные дополнения к браузерам Chrome и Firefox. В качестве передаваемых описаний конечных автоматов выдавались примеры из пакета NuSMV. Тестирование создания файла производилось с помощью модульного тестирования функции \textit{create\_file(source\_code)}. Все тестовые сценарии прошли успешно.