%%%%%%%%%%%%%%%%%%%%%%%%%%%%%%%%%%%%%%%%%%%%%%%%%%%%%%%%%%%%%%%%%%%%%%%%%%%%%%%%
\intro
%%%%%%%%%%%%%%%%%%%%%%%%%%%%%%%%%%%%%%%%%%%%%%%%%%%%%%%%%%%%%%%%%%%%%%%%%%%%%%%%

В ходе разработки программных и аппаратных систем предъявляются высокие требования к надежности. Данные системы характеризуются высокой сложностью, что делает практически невозможным реализацию контроля качества методом тестирования \cite{mironov-book}. В качестве альтернативы, в случае возможности формализации требований с помощью формул математической логики, представляется возможным верификация модели системы (Model checking). 

Model checking -- это метод формальной верификации, основанный на представлении системы в виде математической модели (обычно, структуры Крипке) и формализации требований (обычно, в виде формулы темпоральной логики). 

Среди инструментов формальной верификации стоит выделить NuSMV. Он основан на проекте Cadence SMV, является свободно-распространяемым программным обеспечением и предоставляет широкие возможности по симуляции и верификации моделей \cite{karpov-book}. К недостаткам данной системы относится недостаток поддержки графических интерфейсов пользователя вследствие ориентации на работу с командной строкой. Данный подход сильно ограничивает число пользователей и порог вхождения. 

В рамках данной работы рассмотрены следующие вопросы:

\begin{itemize}
	\item Обзор существующих графических интерфейсов пользователя;
	\item Формулировка требований к разрабатываемому приложению;
	\item Проектирование архитектуры приложения;
	\item Анализ используемых технологий;
	\item Программная реализация приложения;
	\item Определение методов тестирования.
\end{itemize}