%%%%%%%%%%%%%%%%%%%%%%%%%%%%%%%%%%%%%%%%%%%%%%%%%%%%%%%%%%%%%%%%%%%%%%%%%%%%%%%%
\chapter{Постановка задачи}
%%%%%%%%%%%%%%%%%%%%%%%%%%%%%%%%%%%%%%%%%%%%%%%%%%%%%%%%%%%%%%%%%%%%%%%%%%%%%%%%

Основная задача данной выпускной квалификационной работы бакалавра -- разработка веб-ориентированного графического интерфейса пользователя для программного обеспечения, реализующего верификацию методом проверки моделей, NuSMV. Программа должна реализовывать основные функциональные возможности NuSMV. Целью реализуемого проекта является использование в качестве инструмента в академических курсах.

Набор используемых технологий не специфицирован, однако они должны отвечать всем нижеперечисленным требованиям.

\section{Требования}

Программа должна представлять собой веб-приложение, построенное в соответствии с архитектурой "клиент-сервер". Взаимодействие частей приложения должно быть реализовано с помощью протокола HTTP. Ниже приведены требования к функциональным частям приложения.

\section{Клиентская часть приложения}

В качестве клиента должен использоваться веб-браузер Google Chrome версии не ниже 50 или Mozilla Firefox версии не ниже 45. Различия в графическом интерфейсе пользователя, отображающемся в различных браузеров должны быть минимизированы. 

Клиент должен поддерживать графический ввод верифицируемой модели. Для этого необходимо реализовать поле построение модели, реализующее:

\begin{itemize}
	\item добавление состояния,
	\item удаление состояния,
	\item добавление перехода,
	\item удаление перехода,
	\item установка инициализирующего состояния,
	\item перемещение состояния по полю построения.
\end{itemize}

Управление переменными состояния должно быть реализовано в отдельном модуле. Для спецификаций также должно быть выделено отдельное окно. Данные модули должны быть реализованы таким образом, чтобы обеспечить максимальный контроль приложением вводимых пользователем данных. Также необходимо организовать окно, динамически отображающее генерируемый код на языке SMV.

\section{Серверная часть приложения}

Сервер должен принимать и обрабатывать HTTP-запросы клиента, хранить и выдавать статические файлы, а также запускать процесс NuSMV. Также серверная часть должна корректно обрабатывать запросы от нескольких пользователей. Целевой операционной системой является Windows 10.

